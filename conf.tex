\documentclass[12pt,a4paper]{report}

% Paquetes para el manejo de la lengua y la codificación de los caracteres
\usepackage[spanish]{babel} % Para tener los elementos de texto en español
\usepackage[T1]{fontenc} % Set the font (output) encodings for español

% Paquetes para el manejo de gráficos e imágenes
\usepackage{graphicx} % Mejoras sobre el paquete graphics
\graphicspath{ {imagenes/} } % carpeta en la que va a buscar por imágenes

% Paquetes para el manejo de matemáticas
\usepackage{amsmath} % para poner flechas dentro del código

% Paquetes para el manejo de la geometría del documento y la justificación del texto
\usepackage[a4paper,top=2cm,bottom=2cm,left=3cm,right=3cm,marginparwidth=1.75cm]{geometry} % para la portada. REVISAR QUE HACE
\usepackage[protrusion=false, expansion=true]{microtype} % mejora la justificación del documento

% Paquetes para el manejo de hipervínculos
\usepackage[pdfpagelabels,colorlinks=true, allcolors=black]{hyperref} %pone los links en negro

% Paquetes para el manejo de la bibliografía
\usepackage[style=ieee]{biblatex}
\usepackage{csquotes} % Facilitar el trabajo con citas
\addbibresource{export.bib}
\setcounter{biburlnumpenalty}{9000}
\setcounter{biburllcpenalty}{9000} 
\setcounter{biburlucpenalty}{9000} 

\addto\extrasspanish{
    \def\sectionautorefname{Capítulo}
    \def\subsectionautorefname{Apartado}
}

% Paquetes para el manejo de código fuente
\usepackage{listingsutf8}
\usepackage{pxfonts,fix-cm}
\usepackage{accsupp} % Para poder poner los número en el código y que no se copien al seleccionar el código
\usepackage{xcolor}
\newcommand{\noncopynumber}[1]{%
    \BeginAccSupp{method=escape,ActualText={}}%
    #1%
    \EndAccSupp{}%
}

% Configuración de la presentación del código
\definecolor{sviolet}{HTML}{6C71C4}
\definecolor{sbase1}{HTML}{93A1A1}
\definecolor{sblue}{HTML}{268BD2}
\definecolor{scyan}{HTML}{2AA198}
\definecolor{sbase00}{HTML}{657B83}
\lstset{
    inputencoding=utf8/latin1,
    language=Java, 
    columns=fullflexible, % si lo comentas queda algo mas bonito
    keepspaces=true, % si lo comentas queda algo mas bonito
    sensitive=true,
    aboveskip=\baselineskip,
    belowskip=\baselineskip,
    frame=lines,
    xleftmargin=\parindent,
    belowcaptionskip=1\baselineskip,
    basicstyle=\color{sbase00}\ttfamily,
    keywordstyle=\color{scyan},
    commentstyle=\color{sbase1},
    stringstyle=\color{sblue},
    numberstyle=\color{sviolet},
    identifierstyle=\color{sbase00},
    breaklines=true,
    showstringspaces=false,
    tabsize=2
}

\lstdefinestyle{mystyle}{
    basicstyle=\ttfamily\footnotesize,
    breaklines=true,
    columns=fullflexible,
    frame=single,
    captionpos=b,
    keepspaces=true,
    showspaces=false,
    showstringspaces=false,
    numberstyle=\tiny,
    numbers=left,
    stepnumber=1,
    numbersep=5pt
}

% Paquetes para el manejo de la disposición de los elementos en la página
\usepackage{float} % Para centrar las imágenes 
\usepackage{parskip} % si no pones esto, el titulo de la portada sale mal
% \usepackage[all]{hypcap} % Ajusta los enlaces para que apunten a la parte superior de las figuras y tablas
\usepackage[hypcap=true]{caption} % hypcap esta desactualizado
\usepackage{chngcntr} % Desvincula el contador de las figuras del contador de las secciones
\counterwithout{figure}{chapter} % para que la numeración de las figuras no siga la de la sección en la que esta
\usepackage{awesomebox} % Bloques de texto más molones
\setlength{\parskip}{0pt} % definimos la separación entre párrafos
\setlength{\parindent}{20pt} % definimos la identación de los párrafos

% Paquetes para la creación de la portada
\usepackage{tikz} % para crear gráficos y figuras mas complejos. figura azul de portada
\usepackage{changepage} % para editar margenes y otras dimensiones de la pagina
\usepackage{afterpage}% for "\afterpage"

\definecolor{color_29791}{rgb}{0,0,0} % color para la figura de la portada
\definecolor{color_104998}{rgb}{0.290196,0.494118,0.733333} % color para la figura de la portada
\definecolor{AzulCeleste}{RGB}{31, 130, 192} % para la portada


\usepackage{booktabs}
\usepackage{array}
\usepackage{tabularx}
\usepackage{todonotes}

% Paquetes deshabilitados
% \usepackage{pict2e} % mejora del paquete picture. Para cosas mas complejas usar tikz
% \usepackage{wasysym} % da conflictos con amsmath
% \usepackage{latexsym} 
% \usepackage[utf8]{inputenc} % ya no hace falta desde 2018
% \usepackage{pdfpages} % Para insertar pdf en el documento
% \usepackage[none]{hyphenat} %evitamos que rompa las palabras
