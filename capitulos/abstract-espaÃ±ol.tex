Este proyecto es el resultado de una colaboración entre la facultad de INEF y el centro de investigación GAMMA, ambos pertenecientes a la UPM. Consiste en el desarrollo de una aplicación para la visualización de movimientos deportivos capturados mediante un sistema de Motion Capture, en el formato de archivos C3D. 

La aplicación permite reproducir un fichero sobre un entorno tridimensional, pudiendo visualizar el movimiento con diferentes vistas. El programa admite un fichero de configuración para personalizar la visualización, pudiendo variar características de los marcadores. Además, es posible añadir uniones entre marcadores y vectores de velocidad y aceleración. 

Esta aplicación se ha desarrollado integramente en Rust, utilizando únicamente herramientas de este lenguaje.