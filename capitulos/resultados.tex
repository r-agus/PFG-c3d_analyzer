\noindent Este capítulo describe de manera sistemática todas las pruebas a las que fue sometido el diseño o solución propuesta para comprobar su correcto funcionamiento. Al igual que en el capítulo anterior, la descripción de los experimentos ha de ser tal que permita la reproducción de los mismos.

Del mismo modo, hay que incluir los resultados obtenidos de estos experimentos. Sin embargo, no conviene caer en el error de “inundar” el informe de grandes tablas de datos numéricos. Es mejor incluir los resultados ya procesados y sintetizados en pequeñas tablas o gráficas. Si alguna gran tabla de datos es especialmente significativa, se puede incluir en un anexo. Por otro lado, tampoco conviene caer en la redundancia presentando los mismos datos en distintos formatos.

Finalmente, es necesario complementar los resultados con una interpretación de los mismos: ¿por qué salen así?, ¿son coherentes con los supuestos de diseño?, ¿indican la presencia de algún error en la fase de diseño?, ¿corroboran la validez del diseño como solución al problema planteado?, etc.
